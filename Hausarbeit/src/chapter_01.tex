\section{Einleitung und Motivation}

\subsection{Hintergrund und Relevanz}
Machine Learning ist längst kein Zukunftsszenario mehr - es bestimmt bereits heute wichtige Entscheidungen
in unserem Alltag. Netflix empfiehlt uns Inhalte, Banken bewerten Kreditanträge automatisiert, und Amazon
blockiert verdächtige Transaktionen in Echtzeit \cite{netflix_ml}\cite{use-cases-of-ml-in-finance-and-banking}\cite{futuremedesign}.
Die Zahlen sprechen für sich: Laut Eurostat nutzen mittlerweile über 13 Prozent der europäischen Unternehmen KI-Technologien,
und bei großen Firmen sind es sogar 42 Prozent \cite{eurostat_digitalisation_2025}.

Machine Learning wird von IBM als Ansatz beschrieben, bei dem Algorithmen Muster in Daten lernen,
ohne für jede Entscheidung explizit programmiert zu werden \cite{ibm_ml}.
Moderne Kredit-Scoring-Modelle nutzen genau dieses Prinzip, indem sie auf Basis vieler historischer
Kreditanträge komplexe Risikomuster für Zahlungsausfälle lernen \cite{svitla_credit_scoring}.

Die Vorteile sind klar: Prozesse laufen automatisiert rund um die Uhr, Entscheidungen fallen in Sekunden
statt Tagen, und Unternehmen können ihre Kunden viel individueller behandeln.
Das macht ML wirtschaftlich interessant und erklärt den großen Hype um die Technologie.
Schätzungen gehen davon aus, dass die globale Machine-Learning-Industrie bereits 2024
einen Marktwert von rund 80 bis über 200 Milliarden US-Dollar erreicht und in den
kommenden Jahren weiter stark wächst \cite{mlstats_2024}\cite{ml_market_204b}.

Trotz dieser beeindruckenden Vorteile und der großen Investitionen
scheitern viele ML-Projekte, bevor sie überhaupt produktiv gehen.
Mehrere Studien zeigen, dass ein erheblicher Teil der entwickelten
Modelle nie den Schritt in stabile Produktivsysteme schafft und
dass die Ursachen meist in Datenqualität, Infrastruktur und
Organisation liegen - nicht in den Algorithmen selbst
\cite{paleyes2022challenges}\cite{kdnuggets2024mlfail}\cite{qcon2025mlfail}\cite{ihl_80_ai_fail}.

\subsection{Ziel der Arbeit und Aufbau}

