\section{Grundlagen von Machine Learning}

\subsection{Was ist Machine Learning?}
Machine Learning ist ein Teilgebiet der Künstlichen Intelligenz, das sich mit der Entwicklung von Systemen
befasst, die aus Erfahrung lernen und ihre Leistung durch diese Erfahrung verbessern können
\cite{ibm_ml}. Im Kern geht es darum, dass Computer Muster in großen Datenmengen
selbstständig erkennen und diese Muster dann nutzen, um Vorhersagen zu treffen oder Entscheidungen zu treffen
\cite{aws_whatis_ml}.

Technisch unterscheidet man dabei grundsätzlich drei Lernparadigmen. Beim überwachten Lernen (Supervised Learning)
wird das System mit bereits gelabelten Beispielen trainiert - etwa historischen Kreditanträgen mit der
Information, ob ein Ausfall eintrat oder nicht
\cite{svitla_credit_scoring}. Das System lernt, diese Muster
auf neue, ungesehene Daten zu übertragen. Beim unüberwachten Lernen hingegen sucht das System eigenständig
nach Strukturen in Daten ohne vorgegebene Labels, etwa zur Kundensegmentierung
\cite{ibm_ml,konfuzio_ml}. Reinforcement Learning schließlich arbeitet nach dem Prinzip von Belohnung
und Bestrafung - das System optimiert seine Entscheidungen durch Feedback aus der Umgebung
\cite{ibm_ml}.

Das Besondere an Machine Learning ist die Generalisierungsfähigkeit. Ein gut trainiertes Modell kann Muster,
die es in den Trainingsdaten gelernt hat, auf völlig neue, unbekannte Situationen übertragen. Darum kann ein
Betrugserkennungssystem, das mit tausenden historischen Transaktionen trainiert wurde, auch morgen noch
verdächtige Muster in neuen Transaktionen erkennen
\cite{use-cases-of-ml-in-finance-and-banking}.

Im Endeffekt handelt es sich nur um ein statistisch-mathematisches Verfahren, das Korrelationen in Daten
findet. Also: Je besser die Daten, desto besser sind auch die Ergebnisse. Schlechte oder verzerrte Daten führen
zwangsläufig zu schlechten Modellen - ganz unabhängig davon, wie ausgefeilt der Algorithmus ist
\cite{datasolut_ml}. Und genau das unterschätzen viele Unternehmen: die Datenqualität.


\subsection{Typische Herausforderungen in der Praxis}
In der Praxis zeigt sich jedoch ein Paradoxon: Viele Modelle funktionieren im geschützten Laborumfeld hervorragend,
scheitern aber beim Übergang in den produktiven Einsatz. Der häufigste Grund ist nicht mangelnde Algorithmik,
sondern fehlende operative Struktur - unterschiedliche Softwareumgebungen, sich verändernde Daten,
undokumentierte Prozesse und unklar verteilte Verantwortlichkeiten \cite{paleyes2022challenges}.
Um diese Probleme besser zu verstehen, ist es hilfreich, zunächst den Machine-Learning-Lebenszyklus zu betrachten - und
dann zu analysieren, wo konkret Reproduzierbarkeit, Übergabeprozesse und Datendrift zu Problemen führen.