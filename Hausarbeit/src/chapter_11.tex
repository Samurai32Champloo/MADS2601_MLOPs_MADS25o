\section{MLOps Architektur und Technische Komponenten}

Da nun das End-to-End Architekturmodell mit seinen Prozessschritten erläutert wurde, ist es sinnvoll, im nächsten Schritt die technischen Komponenten dieser Architektur näher zu betrachten. 
Dabei ist insbesondere von Interesse, welche Aufgaben sie erfüllen, welche Herausforderungen mithilfe dieser Komponenten adressiert werden, 
in welcher Weise sie miteinander interagieren und welche Anbieter in diesem Zusammenhang relevant sind.
Auf die MLOps als Weiterentwicklung der bekannten DevOps wurde bereits in \fixme{queerverweis auf section{Von DevOps zu MLOps} einbauen} eingegangen. 
In diesem Kapitel werden zunächst grundlegende Komponenten aus dem DevOps Umfeld untersucht, die ebenfalls im Kontext der MLOps Anwendung finden. 
Anschließend wird dargestellt, welche zusätzlichen Komponenten speziell für MLOps erforderlich sind und wie sich diese mit den bestehenden Strukturen integrieren lassen. 

\begin{figure}[!ht]
	\centering
	\includegraphics[width=0.90\textwidth]{src/abbildungen/MLOps_reference_architecture.png}
	\caption{MLOp Referenz Architektur \cite{wozniak2025}}
	\label{fig:MLOps_reference_architecture}
\end{figure}

In Abbildung \ref{fig:MLOps_reference_architecture} ist eine Zusammenfassung zentraler MLOps Komponenten sowie ausgewählter etablierter Lösungen dargestellt, 
die die Beziehung zwischen DevOps und MLOps veranschaulicht \cite{wozniak2025}.
\newpage

\subsection{Klassische DevOps-Komponenten}

\subsubsection{Source Code Repository}

\subsubsection{CI/CD-Komponente}

\subsubsection{Container Orchestrierung}


\subsection{MLOps-spezifische Kernkomponenten}


\subsubsection{Feature Store System}

\subsubsection{Model Registry}

\subsubsection{Trainingsinfrastruktur und Modell-Serving}

\subsubsection{Monitoring-Komponente}