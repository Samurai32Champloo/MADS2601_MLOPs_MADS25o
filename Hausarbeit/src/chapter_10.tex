\section{MLOps als Rahmenwerk für den zuverlässigen Betrieb von ML-Modellen}

Ein ganzheitlicher Blick auf MLOps ist zentral, um die vielfältigen Zusammenhänge zwischen organisatorischen Abläufen,
technischen Komponenten und grundlegenden Prinzipien moderner ML-Systeme zu verstehen. 
MLOps wird dabei als Rahmen betrachtet, der die bestehenden Herausforderungen im Umgang mit datengetriebenen Modellen strukturiert adressiert und zugleich Orientierung für robuste, skalierbare und nachvollziehbare Prozesse bietet. 
Die folgende Ausarbeitung zielt darauf ab, die wesentlichen Prozessschritte, Rollen und architektonischen Bausteine in einer abstrahierten, allgemein übertragbaren Form zusammenzuführen. 
Zugleich bleibt MLOps ein flexibel adaptierbares Konzept, dessen konkrete Umsetzung stets von den spezifischen organisatorischen, technischen und projektbezogenen Rahmenbedingungen abhängt.


\subsection{Das ganzheitliche MLOps-Workflow-Modell}
Der Prozess der Operationalisierung von maschinellen Lernverfahren erfordert eine klar strukturierte Abfolge von Aufgaben,
die sowohl die Automatisierung als auch die Integration von Modellen in produktive Systeme adressieren. 
Ziel von MLOps ist es, manuelle Arbeitsschritte innerhalb von ML-Prozessen zu reduzieren und die Überführung von Proofs of Concept in die Produktion zu erleichtern \cite[S. 31866]{kreuzberger2023}. 
In der Literatur werden vergleichbare Prozessstrukturen beschrieben, die jedoch in Bezeichnung, Schwerpunktsetzung und Integration der einzelnen Schritte variieren. 
Grundlegende Phasen wie Datenvorbereitung, Feature Engineering, Modellentwicklung, Deployment und Monitoring werden in allen Arbeiten behandelt, 
unterscheiden sich jedoch in der Abfolge, Unterteilung in Aufgabenpakete,
der Verzahnung mit DevOps-Praktiken sowie der Einbindung kontinuierlicher Feedback- und Qualitätskontrollmechanismen \cite[S. 31873]{kreuzberger2023}, \cite[S. 22171-22173]{wozniak2025}, \cite[S. 3]{wazir2023}, \cite[S. 3-6]{berberi2025}. 
So betonen Wozniak et al. die klare Trennung zwischen Daten- und Modellvorbereitung, gefolgt von Deployment und Monitoring,
während Wazir et al. eine initiale Anforderungsanalyse vorsehen und die Datensammlung, Datenaufbereitung, Feature Engineering sowie das Training und die Evaluation der Modelle als eine zusammenhängende Phase der Modellentwicklung betrachten.
Aus der Analyse der unterschiedlichen Prozessmodelle lässt sich eine abstrahierte Sicht auf MLOps gewinnen, in der die zentralen Aktivitäten in vier Bereiche zusammengefasst werden können, die Projektinitialisierung, Datenvorbereitung, Modellentwicklung und Workflow-Automatisierung umfassen.
Diese Einteilung bietet einen strukturierten Rahmen, um die unterschiedlichen Prozessmodelle aus der Literatur einzuordnen und die zentralen Zusammenhänge zwischen Aufgaben, Rollen und architektonischen Bausteinen sichtbar zu machen.
Aufbauend auf diesen Überlegungen zeigt die generalisierte MLOps End-to-End Architektur von Kreuzberger et al. den gesamten Ablauf vom Start eines MLOps-Produkts bis zur Modell Bereitstellung \cite[S. 31873]{kreuzberger2023}.
\subsection{Projektinitialisierung und Datenerfassung}

\subsection{Datenvorbereitung und Feature Engineering}

\subsubsection{Anforderungen und iterative Regeldefinition}

\subsubsection{Extraktion, Transformation und Validierung}

\subsection{Modellentwicklung und Experimentierung}

\subsubsection{Modell-Engineering und Hyperparameter-Optimierung}

\subsubsection{Code-Commit, CI/CD-Trigger und Artefakterstellung}

\subsection{Die automatisierte ML-Workflow-Pipeline}

\subsubsection{Automatisches Training, Evaluierung und Registrierung}

\subsubsection{Kontinuierliches Monitoring und Feedback-Schleifen}