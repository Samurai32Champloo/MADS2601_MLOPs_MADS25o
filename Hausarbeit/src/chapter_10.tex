\section{MLOps als Rahmenwerk für den zuverlässigen Betrieb von ML-Modellen}

Ein ganzheitlicher Blick auf MLOps ist zentral, um die vielfältigen Zusammenhänge zwischen organisatorischen Abläufen, technischen Komponenten und grundlegenden Prinzipien moderner ML-Systeme zu verstehen. MLOps wird dabei als Rahmen betrachtet, der die bestehenden Herausforderungen im Umgang mit datengetriebenen Modellen strukturiert adressiert und zugleich Orientierung für robuste, skalierbare und nachvollziehbare Prozesse bietet. Die folgende Darstellung zielt darauf ab, die wesentlichen Prozessschritte, Rollen und architektonischen Bausteine in einer abstrahierten, allgemein übertragbaren Form zusammenzuführen. Zugleich bleibt MLOps ein flexibel adaptierbares Konzept, dessen konkrete Ausgestaltung stets von spezifischen Anforderungen, Ressourcen und organisationalen Rahmenbedingungen geprägt ist.

\subsection{Das ganzheitliche MLOps-Workflow-Modell}

\subsection{Projektinitialisierung und Datenerfassung}

\subsection{Datenvorbereitung und Feature Engineering}

\subsubsection{Anforderungen und iterative Regeldefinition}

\subsubsection{Extraktion, Transformation und Validierung}

\subsection{Modellentwicklung und Experimentierung}

\subsubsection{Modell-Engineering und Hyperparameter-Optimierung}

\subsubsection{Code-Commit, CI/CD-Trigger und Artefakterstellung}

\subsection{Die automatisierte ML-Workflow-Pipeline}

\subsubsection{Automatisches Training, Evaluierung und Registrierung}

\subsubsection{Kontinuierliches Monitoring und Feedback-Schleifen}