\section{Der Machine Learning Lifecycle}

\subsection{Die Phasen des ML-Lebenszyklus}
Machine Learning ist kein einmaliger Prozess, sondern eher ein kontinuierlicher Kreislauf mit mehreren
aufeinanderfolgenden Phasen \cite{kreuzberger2023, wazir2023}. Es beginnt mit der
eigentlichen Problemdefinition und Datensammlung: Welches Business-Problem soll gelöst werden? Welche
Daten sind überhaupt verfügbar oder müssen erhoben werden? \cite{paleyes2022challenges}. Danach folgt die
Explorationsphase: Daten werden gereinigt, analysiert und Features werden engineert - dieser Schritt
ist oft sehr zeitintensiv und iterativ \cite{wozniak2025}. In dieser Phase werden auch erste Hypothesen
getestet und die Qualität der verfügbaren Daten beurteilt - oft zeigt sich erst hier, ob die gesammelten
Daten überhaupt für die ML-Aufgabe geeignet sind \cite{paleyes2022challenges}.

In der Modellierungsphase trainiert der Data Scientist verschiedene Algorithmen, vergleicht die Performance und
optimiert Hyperparameter \cite{paleyes2022challenges}. Das beste Modell wird dann in die Deploymentphase überführt -
es wird containerisiert, in ein Serving-System integriert und dann in eine Produktionsumgebung gebracht \cite{kreuzberger2023}.
Dieser Schritt ist oft kritischer als man denkt: Das Modell muss nicht nur mathematisch funktionieren, sondern auch
in der realen Umgebung mit echten Datenvolumina, echten Latenzanforderungen und echten Fehlerszenarien zuverlässig laufen
\cite{wazir2023}.

In der Betriebsphase läuft das Modell kontinuierlich und macht Vorhersagen. Gleichzeitig
werden seine Performance und die Datenqualität überwacht \cite{wozniak2025, testi2022}. Schließlich gibt es die
Feedbackphase, in der echte Ergebnisse (Ground Truth) mit Vorhersagen verglichen werden \cite{paleyes2022challenges}.
Dieser Zyklus beginnt dann von vorne - entweder durch regelmäßiges Retraining oder durch ereignisgesteuerte
Retraining bei erkanntem Datendrift \cite{karamitsos2020}. Ohne diese Feedbackschleife altert das Modell schnell,
denn die Realität ändert sich ständig: neue Kundengruppen erscheinen, wirtschaftliche Bedingungen verschieben sich,
technische Systeme werden aktualisiert \cite{paleyes2022challenges}.

Das Entscheidende: Dieser Kreislauf sollte nicht manuell, sondern automatisch ablaufen, damit Modelle kontinuierlich
verbessert werden, ohne dass jedes Mal ein Mensch manuell eingreifen muss \cite{john2021}. Das bedeutet konkret:
Datenaufbereitung, Model Training und Evaluation sollten automatisiert werden - ähnlich wie in klassischen
Software-Entwicklungspipelines. Nur so kann gewährleistet werden, dass neue Erkenntnisse schnell ins Modell
einfließen, dass Datenqualitätsprobleme früh erkannt werden, und dass ein kontinuierlicher Verbesserungsprozess
stattfindet, ohne dass jedes Mal manuelle Schritte nötig sind \cite{john2021, karamitsos2020, testi2022}.

\newpage

\begin{figure}[!ht]
	\centering
	\includegraphics[width=0.75\textwidth]{src/abbildungen/mlops-cycle.png}
	\caption{Der MLOps-Zyklus: Entwicklung (links) und Betrieb (rechts) als kontinuierlicher Prozess \cite{databricks_mlops_cycle}}
	\label{fig:mlops_cycle}
\end{figure}

Abbildung \ref{fig:mlops_cycle} visualisiert diesen Kreislauf: Der linke Teil zeigt die Entwicklungsphase 
mit Datenaufbereitung, explorativer Analyse und Training, während der rechte Teil den Betriebszyklus mit 
Deployment, Inferenz und Monitoring darstellt \cite{databricks_mlops_cycle}. Die Review-Phase verbindet beide Kreisläufe und sorgt dafür, 
dass Erkenntnisse aus dem Betrieb zurück in die Entwicklung fließen - ein kontinuierlicher Verbesserungsprozess.

\subsection{Typische Herausforderungen} % Reproduzierbarkeit, Übergabe, Drift ...

