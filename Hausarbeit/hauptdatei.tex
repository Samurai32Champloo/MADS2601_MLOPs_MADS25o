%%%%%%%%%%%%%%%%%%%%%%%%%%%%%%%%%%%%%%%%%%%%%%%%%%%%%%%%%%%%%%%%%%%
%                                                                 %
%                 Packages / Grundeinstellungen                   %
%                                                                 %
%%%%%%%%%%%%%%%%%%%%%%%%%%%%%%%%%%%%%%%%%%%%%%%%%%%%%%%%%%%%%%%%%%%

% Erstellen eines PDF/A 4 
\DocumentMetadata{
    pdfversion=2.0,
    pdfstandard=A-4,
    lang=de,
}

% Festlegung des Allgemeinen Dokumentenformat
\documentclass[a4paper,12pt,parskip=half,headsepline,DIV=12,numbers=noenddot]{scrartcl}

%%%%%% Muss in die documentclass %%%%%%%%
%BCOR=12mm, Korrektur fuer die Bindung
%DIV12 DIV-Wert fuer die Erstellung des Satzspiegels

% Keine floats in andere Sections
\usepackage[section]{placeins}
% Weitere Pakete
\usepackage{microtype}
\usepackage{caption}
\captionsetup[listing]{
  aboveskip=0.25cm, % Abstand direkt vor dem Float
  belowskip=0.2cm  % Abstand direkt nach dem Float
}
\usepackage{pdflscape}
\usepackage{float}
\usepackage{dirtree}
\usepackage{subcaption}
\usepackage{enumitem}
% Booktabs Tabellen
\usepackage{tabularray}
\UseTblrLibrary{booktabs}
\DefTblrTemplate{contfoot-text}{normal}{Fortsetzung auf nächster Seite}
\SetTblrTemplate{contfoot-text}{normal}
\DefTblrTemplate{conthead-text}{normal}{}
\SetTblrTemplate{conthead-text}{normal}

% Um Captions in Tabellen zu deaktivieren 
%\DefTblrTemplate{caption-tag}{default}{}
%\DefTblrTemplate{caption-sep}{default}{}
%\DefTblrTemplate{caption-text}{default}{}

% Grafiken aus PNG Dateien einbinden
\usepackage{graphicx}

% Deutsche Sonderzeichen und Silbentrennung nutzen
\usepackage[ngerman]{babel}
\usepackage{blindtext}

% Eurozeichen einbinden
\usepackage[right]{eurosym}

% Kopf- und Fußzeilen
\usepackage[headsepline,autooneside=false]{scrlayer-scrpage}
\clearpairofpagestyles

% Schriftart 
\usepackage{fontspec}
\setmainfont{TeX Gyre Termes}
\setsansfont{TeX Gyre Adventor}

% Floatende Bilder ermöglichen
\usepackage{floatflt}

% tikz
\usepackage{tikz}
\usetikzlibrary{calc,arrows,math}
\usetikzlibrary{shapes.geometric,positioning}

%% Schaltpläne nach europäischen Richtlinien
\usepackage[european]{circuitikz}
\tikzset{x=1mm,y=1mm}

\usepackage{siunitx}
\sisetup{output-decimal-marker={,},detect-all}

% Bricht lange URLs "schön" um
\usepackage[hyphens,obeyspaces,spaces]{url}

% Paket für Textfarben
\usepackage{xcolor} 
\definecolor{LightGray}{gray}{0.9}
\usepackage[pagecolor=white]{pagecolor}

% Definiere eigene Kommentare
\usepackage{xcolor}
\newcommand{\fixme}[1]{\textcolor{red}{FIXME: #1}}
\newcommand{\todo}[1]{\textcolor{blue}{TODO: #1}}
\newcommand{\note}[1]{\textcolor{gray}{NOTE: #1}}

% Mathematische Symbole importieren
\usepackage{amssymb}

% Paket für Zeilenabstand
\usepackage{setspace}

% Für Bildbezeichner
\usepackage{capt-of}

% Für Stichwortverzeichnis
\usepackage{makeidx}

% Für if und while 
\usepackage{etoolbox}

% Konfiguriere das Inhaltsverzeichnis
\usepackage{tocbasic}
\DeclareTOCStyleEntries[
  raggedentrytext,
  %numwidth=0pt, if numbers=noenddot is not set
  numsep=1ex,
  dynnumwidth,
]{tocline}{chapter,section}
\DeclareTOCStyleEntries[
  linefill=\TOCLineLeaderFill,
]{tocline}{section,subsection,subsubsection,paragraph,subparagraph}

\newcommand*\tocentryformat[1]{{\rmfamily#1}}
\RedeclareSectionCommands
  [tocentryformat=\tocentryformat,tocpagenumberformat=\tocentryformat]{subsection,subsubsection,paragraph,subparagraph}

\newcommand*\tocentrysectionformat[1]{{\rmfamily\bfseries#1}}
\RedeclareSectionCommands
  [tocentryformat=\tocentrysectionformat,tocpagenumberformat=\tocentrysectionformat]{section}  
  
\DeclareTOCStyleEntries[
  pagenumberbox=\hbox,
  dynnumwidth]{tocline}{chapter,section,subsection,subsubsection,paragraph,subparagraph,figure,table}

% Für schönere Listings
\usepackage[newfloat]{minted}
\setminted{
  frame=lines,
  framesep=2mm,
  baselinestretch=1.2,
  bgcolor=LightGray,
  fontsize=\footnotesize,
  linenos,
  breaklines=true,
  breakanywhere=true,
  autogobble,
  tabsize=2,
  listparameters=
    \setlength{\topsep}{0pt}
    \setlength{\partopsep}{0pt}
    \setlength{\parsep}{0pt}
}

%% Mehrzeilige inline Codeblöcke können nur mit bgcolor=none umgesetzt werden.
\setmintedinline{breakanywhere=false, breaklines=true, bgcolor=LightGray}

% Keine Floats bei Listings
\newenvironment{code}[2]
  {\captionsetup{type=listing}
  \providecommand{\captiontitle}{#1}
  \providecommand{\labeltitle}{#2}
  }
  {
  \caption{\captiontitle}
  \label{\labeltitle}
  }
\SetupFloatingEnvironment{listing}{}

% Nummerierung inkl. Section
\usepackage{chngcntr}
\counterwithin{table}{section}
\counterwithin{figure}{section}
\counterwithin{listing}{section}

% Abkürzungsverzeichnis
\usepackage[printonlyused, smaller, withpage]{acronym}

% Erzeugt Inhaltsverzeichnis mit Querverweisen zu den Abschnitten (PDF Version)
\usepackage[bookmarksnumbered,hyperfootnotes=false,hypertexnames=false]{hyperref}
\hypersetup{
    colorlinks=true,
    linkcolor=black,
    filecolor=blue,
    citecolor = black,      
    urlcolor=blue,
}

% Darf erst hier eingebunden werden! 
\usepackage{subfiles}
\usepackage{csquotes}

% Indexerstellung
\makeindex


%%%%%%%%%%%%%%%%%%%%%%%%%%%%%%%%%%%%%%%%%%%%%%%%%%%%%%%%%%%%%%%%%%%
%                                                                 %                    
%                   Definition Zitierstil                         %
%                                                                 %
%%%%%%%%%%%%%%%%%%%%%%%%%%%%%%%%%%%%%%%%%%%%%%%%%%%%%%%%%%%%%%%%%%%

% Zitierung nach IEEE

\usepackage[
backend=biber,
style=ieee,
autocite=inline,
citestyle=numeric,
]{biblatex}
\addbibresource{bibtex/hauptdatei.bib}

% Zitierung nach APA

%\usepackage[
%backend=biber,
%style=apa,
%autocite=inline,
%]{biblatex}
%\addbibresource{bibtex/hauptdatei.bib}

\setcounter{biburllcpenalty}{7000}
\setcounter{biburlucpenalty}{8000}
%%%%%%%%%%%%%%%%%%%%%%%%%%%%%%%%%%%%%%%%%%%%%%%%%%%%%%%%%%%%%%%%%%%
%                                                                 %                    
%                    Definition Deckblatt                         %
%                                                                 %
%%%%%%%%%%%%%%%%%%%%%%%%%%%%%%%%%%%%%%%%%%%%%%%%%%%%%%%%%%%%%%%%%%%

% true für Masterarbeit / false für Hausarbeit
\newbool{masterarbeit}
\setbool{masterarbeit}{false}

% Setze Studiengang
\newcommand{\department}{Applied Data Science}

% Setze akademischen Grad
\newcommand{\studyprogram}{Master of Science}

% Setze Modulname (Masterarbeit muss false sein)
\newcommand{\modulname}{Data-Science-Projektorganisation}

% Setze Dozent:in (Masterarbeit muss false sein)
\newcommand{\auditor}{\textbf{Dozent:in:} \> Prof. Dr. Michael Schulz}

% Setze Gutachter:innen (Masterarbeit muss true sein)
\newcommand{\firstauditor}{\textbf{Erstgutachter:} \> Prof. Dr. Maxi Mustermann}
\newcommand{\secondauditor}{\textbf{Zweitgutachterin:} \> Prof. Dr. Maxi Musterfrau}

% Setze Titel und Untertitel der Abreit 
\newcommand{\thetitle}{MLOps}
\newcommand{\thesubtitle}{Machine Learning Operations}

% Setze Autor:in und MatNr.
\newcommand{\theauthor}{Maxim Ziegler}
\newcommand{\matriculationnumber}{15019}

% Abstand zwischen Name und MatNr. (siehe Deckblatt)
\newcommand{\myspace}{1.0cm}

% Muss in src/basic_structure/deckblatt.tex einkommentiert werden! 

\newcommand{\secondauthor}{\> Oke Pahl \> MatNr. 14965\\}
\newcommand{\thirdauthor}{\> Maxi Mustermensch \> MatNr. 00000\\}
\newcommand{\fourthauthor}{\> Maxi Mustermensch \> MatNr. 00000\\}
\newcommand{\fifthauthor}{\> Maxi Mustermensch \> MatNr. 00000\\}

% PDF Metadaten
%\hypersetup{pdfinfo={
%Title={\thetitle},
%Author={\theauthor}
%}}

\hypersetup{pdfinfo={
Title={\thetitle},
Author={\theauthor}
}}

%%%%%%%%%%%%%%%%%%%%%%%%%%%%%%%%%%%%%%%%%%%%%%%%%%%%%%%%%%%%%%%%%%%
%                                                                 %                    
%                     Beginn des Inhalts                          %
%                                                                 %
%%%%%%%%%%%%%%%%%%%%%%%%%%%%%%%%%%%%%%%%%%%%%%%%%%%%%%%%%%%%%%%%%%%

%%%%%%%%%%%%%%%%%%%%%%%%%%%%%%%%%%%%%%%%%%%%%%%%%%%%%%%%%%%%%%%%%%%
%  Special Characters:                                            %
%                                                                 %
%             \& \% \$ \# \_ \{ \}                                %
%             \textasciitilde (~)                                 %
%             \textasciicircum (^)                                %     
%             \textbackslash (\)                                  %                    
%      \glqq Text\grqq{} für Anführungszeichen                    %
%%%%%%%%%%%%%%%%%%%%%%%%%%%%%%%%%%%%%%%%%%%%%%%%%%%%%%%%%%%%%%%%%%%

\begin{document}



% Definition Header Sections sollen in der Kopfzeile stehen; Kopfzeile mit Unterstrich
\automark[subsection]{section}
\KOMAoptions{headsepline=true}
%\ihead{Kopfzeile innen}
%\chead{Kopfzeile außen}
\ohead{\headmark}

% Definition footer \pagemark steht für Seitennummer
%\ifoot{Fußzeile innen}
%\cfoot{Fußzeile Mitte}
\ofoot{\pagemark}

% Hier werden die Trennvorschläge inkludiert
\input{src/basic_structure/trennung.tex}

% Leere Seite am Anfang
%\thispagestyle{empty} % erzeugt Seite ohne Kopf- / Fusszeile
%\mbox{}
%\newpage

% Titelseite 
%%%%%%%%%%%%%%%%%%%%%%%%%%%%%%%%%
%           Deckblatt           %
%%%%%%%%%%%%%%%%%%%%%%%%%%%%%%%%%
\setmainfont{TeX Gyre Adventor}
\thispagestyle{empty}
\begin{figure}[h!]
	\centering
	\includegraphics[width=0.6\textwidth]{src/abbildungen/logo.png}
\end{figure}
\begin{center}
	\large{\textbf{\department}}\\
	\large{\textbf{\studyprogram}}\\
	\vspace{1cm}
	\ifbool{masterarbeit}{
		\LARGE{\textbf{Masterarbeit}}\\
		\large{zur Erlangung des akademischen Grades \\ Master of Science}\\
	}
	{
		\large{\textbf{Modul\\ \modulname}}\\
	}
	\vspace*{\fill}
	\line(1,0){450}\\
	\doublespacing
	\textbf{\Large{\thetitle}}\\
	\textbf{\large{\thesubtitle}}\\
	\line(1,0){450}\\
\end{center}
\vspace*{\fill}
\onehalfspacing
\small{
\begin{flushleft}
	\begin{tabbing}
		\textbf{Vorgelegt von:} \hspace*{0.8cm}\= \theauthor \hspace*{\myspace}\= MatNr. \matriculationnumber \\

		%%%%%%%%%%%%%%%%%%%%%%%%%%%%%%%%%%%%%%%%%%%%%%
		%                                            %
		%  Hier weitere Autor:innen einkommentieren  %
		%   Müssen hauptdatei.tex definiert sein     %
		%	                                         %
		%%%%%%%%%%%%%%%%%%%%%%%%%%%%%%%%%%%%%%%%%%%%%%		
		\secondauthor
		%\thirdauthor
		%\fourthauthor
		%\fifthauthor

		\textbf{Vorgelegt am:} \> \today\\
		\ifbool{masterarbeit}{
			\firstauditor\\
			\secondauditor\\
		}
		{
			\auditor\\
		}
	\end{tabbing}
\end{flushleft}}
\setmainfont{TeX Gyre Termes}
\newpage

% Singlespacing (Zeilenabstand) (Default)
\singlespacing
\normalsize

% Abstract falls gewünscht
%\thispagestyle{empty}
%\input{abstract}
%\newpage

% Inhaltsverzeichnis anzeigen
\pagestyle{empty}
\tableofcontents
\newpage
\pagestyle{headings}

% Header für den Inhalt 
\KOMAoptions{headsepline=true}
\ohead{\headmark}

% Input Inhalt
\section{Einleitung und Motivation}

\subsection{Hintergrund und Relevanz}
Machine Learning ist längst kein Zukunftsszenario mehr - es bestimmt bereits heute wichtige Entscheidungen
in unserem Alltag. Netflix empfiehlt uns Inhalte, Banken bewerten Kreditanträge automatisiert, und Amazon
blockiert verdächtige Transaktionen in Echtzeit \cite{netflix_ml}\cite{use-cases-of-ml-in-finance-and-banking}\cite{futuremedesign}.
Die Zahlen sprechen für sich: Laut Eurostat nutzen mittlerweile über 13 Prozent der europäischen Unternehmen KI-Technologien,
und bei großen Firmen sind es sogar 42 Prozent \cite{eurostat_digitalisation_2025}.

Machine Learning wird von IBM als Ansatz beschrieben, bei dem Algorithmen Muster in Daten lernen,
ohne für jede Entscheidung explizit programmiert zu werden \cite{ibm_ml}.
Moderne Kredit-Scoring-Modelle nutzen genau dieses Prinzip, indem sie auf Basis vieler historischer
Kreditanträge komplexe Risikomuster für Zahlungsausfälle lernen \cite{svitla_credit_scoring}.

Die Vorteile sind klar: Prozesse laufen automatisiert rund um die Uhr, Entscheidungen fallen in Sekunden
statt Tagen, und Unternehmen können ihre Kunden viel individueller behandeln.
Das macht ML wirtschaftlich interessant und erklärt den großen Hype um die Technologie.
Schätzungen gehen davon aus, dass die globale Machine-Learning-Industrie bereits 2024
einen Marktwert von rund 80 bis über 200 Milliarden US-Dollar erreicht und in den
kommenden Jahren weiter stark wächst \cite{mlstats_2024}\cite{ml_market_204b}.

Trotz dieser beeindruckenden Vorteile und der großen Investitionen
scheitern viele ML-Projekte, bevor sie überhaupt produktiv gehen.
Mehrere Studien zeigen, dass ein erheblicher Teil der entwickelten
Modelle nie den Schritt in stabile Produktivsysteme schafft und
dass die Ursachen meist in Datenqualität, Infrastruktur und
Organisation liegen - nicht in den Algorithmen selbst
\cite{paleyes2022challenges}\cite{kdnuggets2024mlfail}\cite{qcon2025mlfail}\cite{ihl_80_ai_fail}.

\subsection{Ziel der Arbeit und Aufbau}


\newpage
\section{Grundlagen und Projektkontext von Machine Learning}

\subsection{Was ist Machine Learning?}


\subsection{Häufige Praxisprobleme und PoC-Sackgassen}


\newpage
\section{Machine-Learning-Lifecycle und Herausforderungen}

\subsection{Der ML-Lifecycle als Kreislauf}


\subsection{Reproduzierbarkeit und Versionierung}


\subsection{Übergabe zwischen Data Science und IT}


\subsection{Model- und Data-Drift}


\subsection{Monitoring- und Feedback-Lücken}


\subsection{Team-, Kommunikations- und Kulturthemen}


\newpage
\section{Warum ML-Projekte scheitern}

\subsection{Technische Barrieren}


\subsection{Organisatorische Hürden}


\subsection{Fehlendes Monitoring}


\newpage
\section{MLOps als konzeptionelle Lösung}

\subsection{Automatisierung von Workflows}


\subsection{Nachvollziehbarkeit und Lineage}


\subsection{Monitoring und kontinuierliche Verbesserung}


\subsection{Zusammenarbeit und Governance}


\newpage
\section{MLOps als Lösungsansatz}

\subsection{Automatisierung}


\subsection{Nachvollziehbarkeit und Monitoring}


\subsection{Zusammenarbeit}


\newpage
\section{Diskussion und Fazit}

\subsection{Beitrag von MLOps zur Lösung der identifizierten Hürden}


\subsection{Verbleibende Herausforderungen und Ausblick}


\newpage
\section{MLOps als Rahmenwerk für den zuverlässigen Betrieb von ML-Modellen}

Ein ganzheitlicher Blick auf MLOps ist zentral, um die vielfältigen Zusammenhänge zwischen organisatorischen Abläufen,
technischen Komponenten und grundlegenden Prinzipien moderner ML-Systeme zu verstehen. 
MLOps wird dabei als Rahmen betrachtet, der die bestehenden Herausforderungen im Umgang mit datengetriebenen Modellen strukturiert adressiert und zugleich Orientierung für robuste, skalierbare und nachvollziehbare Prozesse bietet. 
Die folgende Ausarbeitung zielt darauf ab, die wesentlichen Prozessschritte, Rollen und architektonischen Bausteine in einer abstrahierten, allgemein übertragbaren Form zusammenzuführen. 
Zugleich bleibt MLOps ein flexibel adaptierbares Konzept, dessen konkrete Umsetzung stets von den spezifischen organisatorischen, technischen und projektbezogenen Rahmenbedingungen abhängt.


\subsection{Das ganzheitliche MLOps-Workflow-Modell}
Der Prozess der Operationalisierung von maschinellen Lernverfahren erfordert eine klar strukturierte Abfolge von Aufgaben,
die sowohl die Automatisierung als auch die Integration von Modellen in produktive Systeme adressieren. 
Ziel von MLOps ist es, manuelle Arbeitsschritte innerhalb von ML-Prozessen zu reduzieren und die Überführung von Proofs of Concept in die Produktion zu erleichtern \cite{kreuzberger2023}. 
In der Literatur werden vergleichbare Prozessstrukturen beschrieben, die jedoch in Bezeichnung, Schwerpunktsetzung und Integration der einzelnen Schritte variieren. 
Grundlegende Phasen wie Datenvorbereitung, Feature Engineering, Modellentwicklung, Deployment und Monitoring werden in allen Arbeiten behandelt, 
unterscheiden sich jedoch in der Abfolge, Unterteilung in Aufgabenpakete,
der Verzahnung mit DevOps-Praktiken sowie der Einbindung kontinuierlicher Feedback- und Qualitätskontrollmechanismen \cite{kreuzberger2023}\cite{wozniak2025}\cite{wazir2023}\cite{berberi2025}. 
So betonen Wozniak et al. die klare Trennung zwischen Daten- und Modellvorbereitung, gefolgt von Deployment und Monitoring,
während Wazir et al. eine initiale Anforderungsanalyse vorsehen und die Datensammlung, Datenaufbereitung, Feature Engineering sowie das Training und die Evaluation der Modelle als eine zusammenhängende Phase der Modellentwicklung betrachten.
Aus der Analyse der unterschiedlichen Prozessmodelle lässt sich eine abstrahierte Sicht auf MLOps gewinnen, in der die zentralen Aktivitäten in vier Bereiche zusammengefasst werden können, die Projektinitialisierung, Datenvorbereitung, Modellentwicklung und Workflow-Automatisierung umfassen.
Diese Einteilung bietet einen strukturierten Rahmen, um die unterschiedlichen Prozessmodelle aus der Literatur einzuordnen und die zentralen Zusammenhänge zwischen Aufgaben, Rollen und architektonischen Bausteinen sichtbar zu machen.
Aufbauend auf diesen Überlegungen zeigt die generalisierte MLOps End-to-End Architektur von Kreuzberger et al. den gesamten Ablauf vom Start eines MLOps-Produkts bis zur Modell Bereitstellung \cite{kreuzberger2023}.
\subsection{Projektinitialisierung und Datenerfassung}
Wozniak et al. haben in einer Literaturanalyse verschiedener MLOps Prozessmodelle identifiziert und analysiert,
welche Aufgabenpakete sich in der Literatur häufig wiederfinden. Dabei finden sie zunächst eine einleitende Stufe,
die sich mit der Analyse des Geschäftsproblems befasst. Obwohl die untersuchten Veröffentlichungen diesen Schritt nicht immer vertiefen,
zeigt die Übersicht dennoch eindeutig, dass eine betriebliche Problem und Zielanalyse in nahezu allen MLOps Ansätzen berücksichtigt wird \cite{wozniak2025}.

Alle der betrachteten Veröffentlichungen beziehen sich in den jeweiligen MLOPs Ansätzen als Grundlage auf den Cross Industry Standard Process for Data Mining CRISP DM oder den Team Data Science Process TDSP. 
Auch CRISP DM beginnt mit einer initialen Phase des Geschäftsverständnisses, in der Ziele bestimmt, Rahmenbedingungen geklärt und Projektziele definiert werden \cite{karamitsos2020}. 
Obwohl CRISP DM nicht speziell für MLOps entwickelt wurde, macht es bereits an zentraler Stelle die Notwendigkeit einer strukturierten Initialisierung sichtbar. 
TDSP führt diese Logik fort und erweitert sie um klar definierte Stakeholder und messbare Erfolgsindikatoren \cite{karamitsos2020}. 
Beide Methoden zeigen, dass datengestützte Projekte traditionell mit einer präzisen Klärung von Zielen und Anforderungen beginnen.

Testi et al. entwickeln auf Basis einer Literaturrecherche einen MLOps Ansatz, der die Phase des Geschäftsverständnisses ebenfalls an den Anfang stellt \cite{testi2022}. 
Diese Phase umfasst die Sammlung und Dokumentation von Anforderungen, die Bestimmung messbarer Erfolgsindikatoren sowie die Ermittlung relevanter Daten \fixme{Sagt im grund das gleich wie TDSP vlt nochmal erwähnen hier}. 
Früh festgelegte Ziele und geeignete Metriken bilden später die Grundlage für das Monitoring produktiver ML Systeme. 
Die Bedeutung der Projektinitialisierung erstreckt sich hier deutlich über die reine Vorbereitung hinaus und wirkt direkt auf die spätere Betriebsphase.

Eine technisch orientierte Ergänzung liefern Bachinger et al. \cite{bachinger2024}. Sie zeigen, dass ein MLOps Prozess nicht nur durch ein geschäftliches Problem ausgelöst werden kann, 
sondern auch durch technische Signale innerhalb einer automatisierten Pipeline. 
Die Identifikation neuer Daten, Strukturänderungen in vorhandenen Daten oder externe Aktualisierungen können den Workflow automatisch anstoßen. 
Dadurch erhält die Projektinitialisierung eine dynamische Komponente, die eng mit Prinzipien kontinuierlicher Integration und Bereitstellung verbunden ist.

Die Zusammenführung aller genannten Quellen zeigt ein konsistentes Muster. 
Jede Form eines datengetriebenen Projekts beginnt mit der Klärung des zugrunde liegenden Problems, 
der Definition der Ziele und der frühen Bewertung der verfügbaren Daten. 
Die generalisierte Architektur nach Kreuzberger et al. erweitert diese etablierten Prinzipien um Anforderungen an Automatisierung, Skalierbarkeit und Betriebssicherheit. 
Dadurch entsteht eine systematische Verbindung zwischen fachlicher Zielsetzung und technischer Umsetzung, die das Fundament für stabile und reproduzierbare ML Systeme bildet.

Kreuzberger et al. beschreiben diese Projektinitialisierung als Startpunkt der gesamten generalisierte MLOps End-to-End Architektur. 
Zunächst wird ein Geschäftsproblem identifiziert und analysiert. Anschließend entsteht eine Lösungsarchitektur, auf deren Grundlage das konkrete ML Problem präzise definiert wird. 
Danach klären Data Engineers und Data Scientists gemeinsam den Datenbedarf, prüfen verfügbare Quellen, bewerten deren Qualität und kontrollieren die Existenz geeigneter Zielvariablen.
Diese Aufgaben erzeugen die fachliche und technische Basis für die spätere Modellierung und bereiten den stabilen Produktivbetrieb vor\cite{kreuzberger2023}.

\subsection{Datenvorbereitung und Feature Engineering}

\subsubsection{Anforderungen und iterative Regeldefinition}

\subsubsection{Extraktion, Transformation und Validierung}

\subsection{Modellentwicklung und Experimentierung}

\subsubsection{Modell-Engineering und Hyperparameter-Optimierung}

\subsubsection{Code-Commit, CI/CD-Trigger und Artefakterstellung}

\subsection{Die automatisierte ML-Workflow-Pipeline}

\subsubsection{Automatisches Training, Evaluierung und Registrierung}

\subsubsection{Kontinuierliches Monitoring und Feedback-Schleifen}
\newpage
\section{MLOps Architektur und Technische Komponenten}

Da nun das End-to-End Architekturmodell mit seinen Prozessschritten erläutert wurde, ist es sinnvoll, im nächsten Schritt die technischen Komponenten dieser Architektur näher zu betrachten. 
Dabei ist insbesondere von Interesse, welche Aufgaben sie erfüllen, welche Herausforderungen mithilfe dieser Komponenten adressiert werden, 
in welcher Weise sie miteinander interagieren und welche Anbieter in diesem Zusammenhang relevant sind.
Auf die MLOps als Weiterentwicklung der bekannten DevOps wurde bereits in \fixme{queerverweis auf section{Von DevOps zu MLOps} einbauen} eingegangen. 
In diesem Kapitel werden zunächst grundlegende Komponenten aus dem DevOps Umfeld untersucht, die ebenfalls im Kontext der MLOps Anwendung finden. 
Anschließend wird dargestellt, welche zusätzlichen Komponenten speziell für MLOps erforderlich sind und wie sich diese mit den bestehenden Strukturen integrieren lassen. 

\begin{figure}[!ht]
	\centering
	\includegraphics[width=0.90\textwidth]{src/abbildungen/MLOps_reference_architecture.png}
	\caption{MLOp Referenz Architektur \cite{wozniak2025}}
	\label{fig:MLOps_reference_architecture}
\end{figure}

In Abbildung \ref{fig:MLOps_reference_architecture} ist eine Zusammenfassung zentraler MLOps Komponenten sowie ausgewählter etablierter Lösungen dargestellt, 
die die Beziehung zwischen DevOps und MLOps veranschaulicht \cite{wozniak2025}.
\newpage

\subsection{Klassische DevOps-Komponenten}

\subsubsection{Source Code Repository}

\subsubsection{CI/CD-Komponente}

\subsubsection{Container Orchestrierung}


\subsection{MLOps-spezifische Kernkomponenten}


\subsubsection{Feature Store System}

\subsubsection{Model Registry}

\subsubsection{Trainingsinfrastruktur und Modell-Serving}

\subsubsection{Monitoring-Komponente}
\newpage
\section{Rollen}

\subsection{Data Scientist, Data Engineer und Software Engineer}


\subsection{ML Engineer/MLOps Engineer und DevOps Engineer}


\newpage
\section{Das MLOps Maturity Model (Reifegradmodell)}





\subsection{Stufen der MLOps-Reife}


\subsubsection{Ad hoc}

\subsubsection{DataOps}

\subsubsection{Manual MLOps}

\subsubsection{Automated MLOps}

\subsubsection{Kaizen}


\subsection{Synthese: Prozess, Architektur und Rollen als ganzheitliches MLOps-Framework}


\newpage
\section{MLOps als Lösungsansatz}

\subsection{Automatisierung}


\subsection{Nachvollziehbarkeit und Monitoring}


\subsection{Zusammenarbeit}


\newpage
\section{Diskussion und Fazit}

\subsection{Beitrag von MLOps zur Lösung der identifizierten Hürden}


\subsection{Verbleibende Herausforderungen und Ausblick}


\newpage

% Literaturverzeichnis anzeigen
\ohead{Literaturverzeichnis} % Korrektur für Header 
\phantomsection
\addcontentsline{toc}{section}{Literaturverzeichnis}
\renewcommand\refname{Literaturverzeichnis}
\printbibliography
\newpage

% Abbildungsverzeichnis anzeigen
\ohead{\headmark}
\listoffigures
\addcontentsline{toc}{section}{Abbildungsverzeichnis}
\newpage


% Tabellenverzeichnis anzeigen
\listoftables
\addcontentsline{toc}{section}{Tabellenverzeichnis}
\newpage


% Listingverzeichnis anzeigen
% \renewcommand{\listlistingname}{Listingverzeichnis}
% \listoflistings 
% \addcontentsline{toc}{section}{Listingverzeichnis}
% \newpage


% Abkürzungsverzeichnis anzeigen
\ohead{Abkürzungsverzeichnis} % Korrektur für Header 
\section*{Abkürzungsverzeichnis}
\input{src/basic_structure/abkuerzungen.tex}
\addcontentsline{toc}{section}{Abkürzungsverzeichnis}
\newpage


% Kein Header für Anhang (Deckblatt) 
\KOMAoptions{headsepline=false}
\ohead{}

% Beginn Anhang
\input{src/anhang/anhang_deckblatt.tex}

% Anhang römisch 
\renewcommand{\thesection}{\Roman{section}}
\renewcommand{\thesubsection}{\Roman{subsection}}
\setcounter{section}{0}
\counterwithin{table}{subsection}
\counterwithin{figure}{subsection}
\counterwithin{listing}{subsection}

% Header Anhang (Inhalt)
\KOMAoptions{headsepline=true}
\ohead{\headmark}
\automark{subsection}

% Input Anhang 
%\input{src/anhang/anhang.tex}
\newpage

% Selbstständigkeits Erklärung
\phantomsection
\addcontentsline{toc}{section}{Selbstständigkeitserklärung}

% Header für Erklärung
\ohead{Selbstständigkeitserklärung}

% Input Erklärung
\input{src/basic_structure/erklaerung.tex}

% Leere Abschlussseite
%\newpage
%\thispagestyle{empty} % erzeugt Seite ohne Kopf- / Fusszeile
%\mbox{}

\end{document}
